\newglossaryentry{off-by-one}
{
    name={off-by-one},
    description={Ein Fehler in der Programmierung, bei dem eine Schleife oder ein Array um eine Einheit zu viel oder zu wenig iteriert oder indiziert wird, was oft zu unerwartetem Verhalten führt. Dies passiert häufig bei der Implementierung von Schleifen oder bei der Arbeit mit Indizes, wo die korrekte Anzahl der Iterationen oder die korrekten Grenzen leicht übersehen werden können}
}

\newglossaryentry{borrow-checker}
{
    name={borrow checker},
    description={Ein System in der Programmiersprache Rust, das zur Kompilierzeit überprüft, ob alle Speicherzugriffe sicher sind. Es stellt sicher, dass Daten nie gleichzeitig sowohl mutabel als auch immutabel geborgt werden, und dass keine doppelten mutablen Referenzen existieren, um Race Conditions und andere Speicherfehler zu verhindern}
}


\newglossaryentry{wrap-around}{
  name={Wrap-around},
  description={Der Begriff "Wrap-around" beschreibt ein Phänomen, bei dem ein numerischer Wert, der seinen maximal oder minimal zulässigen Wert erreicht, auf den gegenüberliegenden Extremwert zurückgesetzt wird. Dies tritt häufig in der Computerprogrammierung auf, insbesondere bei der Arbeit mit beschränkten Datentypen wie Ganzzahlen. Ein Wrap-around kann zu Fehlern führen, wenn es nicht ordnungsgemäß gehandhabt wird.}
}


\newglossaryentry{panic}{
    name={panic},
    description={In der Programmiersprache Rust und anderen Programmiersprachen bezeichnet \textit{panic} eine Laufzeitausnahme, die auftritt, 
    wenn das Programm auf einen schwerwiegenden Fehler stößt, von dem es sich nicht erholen kann. 
    Ein panic führt in der Regel dazu, dass das Programm sofort abgebrochen wird und die Kontrolle an das Betriebssystem zurückgegeben wird. 
    Dies unterscheidet sich von regulären Fehlern, die durch Rückgabewerte oder Ausnahmen behandelt werden können.
    }
}

\newglossaryentry{exception}{
    name={exception},
    description={
        Eine \textit{exception} (dt. Ausnahme) ist ein Mechanismus in vielen Programmiersprachen, der verwendet wird, um auf ungewöhnliche oder fehlerhafte Zustände während der Programmausführung zu reagieren. Ausnahmen treten auf, wenn ein Programm auf ein Problem stößt, das es nicht mit normalem Kontrollfluss handhaben kann, wie z.B. Division durch Null oder der Versuch, auf eine Datei zuzugreifen, die nicht existiert. Beim Auftreten einer Ausnahme wird der normale Ablauf des Programms unterbrochen und es wird ein spezieller Fehlerbehandlungsroutine aufgerufen.
    }
}

\newglossaryentry{data-race}{
  name={Datenrennen},
  description={Ein Datenrennen tritt auf, wenn zwei oder mehr Threads in einem Computerprogramm gleichzeitig auf dieselbe Speicherstelle zugreifen, und mindestens einer der Zugriffe schreibend erfolgt, ohne dass die Zugriffe synchronisiert werden. Dies kann zu unerwarteten und fehlerhaften Verhaltensweisen führen, da die Reihenfolge der Zugriffe nicht garantiert ist und das Ergebnis von der Timing-Synchronisation der Threads abhängt.}
}

\newglossaryentry{tupel}{
  name={Tupel},
  description={Ein Tupel ist eine geordnete Liste von Elementen, die in der Informatik häufig verwendet wird, um eine feste Anzahl von Werten zu speichern, die zu einer Einheit zusammengefasst sind. Die Elemente eines Tupels können unterschiedliche Datentypen haben. Tupel sind unveränderlich, was bedeutet, dass ihre Inhalte nach der Erstellung nicht mehr geändert werden können.}
}


\glsaddall
