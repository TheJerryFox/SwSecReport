\chapter{Zero-Cost-Abstraktionen}

Zero-Cost-Abstraktionen sind ein zentrales Konzept in Rust, das es ermöglicht, hohe Abstraktionen zu verwenden, ohne die Laufzeit des Programms zu beeinträchtigen. 
Dies bedeutet, dass der Overhead, der durch die Abstraktionen entsteht, zur Kompilierzeit eliminiert wird, sodass der generierte Maschinencode ähnlich effizient ist wie handgeschriebener Low-Level-Code \cite{sequeira2023}. 
In anderen Sprachen kann dies zu einem Laufzeit-Overhead führen \cite{haberman2014}.

\section{Iteratoren}
Iteratoren helfen dabei, über eine Menge von Elementen zu iterieren und ersetzen eine explizite Schleife. 
Sie bieten eine abstrahierte Möglichkeit zur Verarbeitung von Sequenzen, wodurch die Programmlogik klarer und fehlerresistenter wird.
\begin{lstlisting}[language=Rust, caption=Iterator]
use std::sync::mpsc;
use std::thread;

pub fn iterator(){
    let numbers = vec![1, 2, 3, 4, 5];

    // Using an iterator to iterate through the vector safely
    for num in numbers.iter() {
        println!("{}", num); // Safely prints each number
    }
}
\end{lstlisting}
Ein solcher Iterator löst die Programmlogik von der Schleifenlogik, wodurch typische Fehler wie \textit{\gls{off-by-one}} und daraus resultierende, z.B. falsche Array-Indexierung (Buffer-Overflow), vermieden werden. 
Darüber hinaus bieten Iteratoren eine einheitliche und oft effizientere Möglichkeit, mit Datenstrukturen zu arbeiten. 
\section{Filter-Methode}

Die \texttt{filter}-Methode ermöglicht es, Elemente eines Iterators basierend auf einer Bedingung zu filtern. 
Dies ergänzt die Funktionalität von Iteratoren und hilft, spezifische Elemente aus einer Sequenz herauszufiltern.
\begin{lstlisting}[language=Rust, caption=Filter-Funktion]
pub fn filter(){
    let numbers = vec![1, 2, 3, 4, 5, 6, 7, 8, 9, 10];

    // Use filter to find even numbers
    let even_numbers: Vec<_> = numbers.iter()
        .filter(|&&x| x % 2 == 0)
        .collect();

    println!("Even numbers: {:?}", even_numbers);
}
\end{lstlisting}
Ein Filter auf einem Iterator separiert zusätzlich die Filterlogik. 
An dieser Stelle könnte auch eine Methode verwendet werden, was für besser lesbaren Code sorgt. 
Die Kombination von \texttt{filter} mit anderen Iterator-Methoden erlaubt eine deklarative Beschreibung von Datenverarbeitungsabläufen, was die Verständlichkeit und Wartbarkeit des Codes weiter erhöht.

\section{Message Passing}

Message Passing ist eine sichere und effiziente Methode zur Kommunikation zwischen Threads.

\begin{lstlisting}[language=Rust, caption=Message Passing ]
pub fn message_passing(){
    let (tx, rx) = mpsc::channel();

    let sender_thread = thread::spawn(move || {
        let msg = "Hello from the sender thread";
        // Sends a message safely to the receiver
        tx.send(msg).unwrap(); 
        println!("Sent message: '{}'", msg);
    });

    let receiver_thread = thread::spawn(move || {
        // Receives message safely
        let received = rx.recv().unwrap(); 
        println!("Received message: '{}'", received);
    });

    sender_thread.join().unwrap();
    receiver_thread.join().unwrap();
}
\end{lstlisting}
Das Beispiel \ref{list:channel} zeigt, wie mit \texttt{mpsc::channel()} ein Sender- und Empfängerkanal erstellt wird, über den Nachrichten sicher zwischen Threads, mittels \texttt{.send()} und \texttt{.recv()}, ausgetauscht werden können.
Durch eine solche einfache Abstraktion können Datenrennen zwischen Threads vermieden werden. 
Trotzdem können Threads effektiv kommunizieren und der Code bleibt lesbar.
\cleardoublepage
\section{Option Typ}

Der \texttt{Option}-Typ wird verwendet, um die mögliche Abwesenheit eines Wertes sicher zu handhaben. Hier ist ein Beispiel, wie man eine Division sicher durchführen kann:

\begin{lstlisting}[language=Rust, caption=Option Typ, label=list:channel]
pub fn option_type(){
    let dividend = 10;
    let divisor = 0;

    match find_divisor(dividend, divisor) {
        Some(result) => println!("Result of division: {}", result),
        None => println!("Cannot divide by zero!"),
    }
}

fn find_divisor(dividend: i32, divisor: i32) -> Option<i32> {
    if divisor == 0 {
        None // Safely handle division by zero
    } else {
        Some(dividend / divisor) // Safely return the result
    }
}
\end{lstlisting}
\noindent
Eine \texttt{Option} gibt entweder einen Wert \texttt{Some} zurück oder einen \texttt{None}-Wert. Mittels \texttt{match} kann geprüft werden, ob ein Wert vorliegt und dementsprechend verarbeitet werden.\\
\\
Dadurch wird vermieden, wie es oft in C++ der Fall ist, dass spezielle Rückgabewerte wie \texttt{-1} oder andere verwendet werden müssen, um besondere Zustände zu signalisieren. 
Ein bekanntes Beispiel dafür ist die Methode \texttt{std::string::find} aus der C++ Standardbibliothek, die \texttt{std::string::npos} (typischerweise \texttt{-1}) zurückgibt, wenn ein Wert nicht gefunden wird. 
Solche Werte können unter Umständen zu Konflikten führen \cite{cppreference_string_find} \cite{cppreference_string_npos}. 
