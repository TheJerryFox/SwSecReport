 \chapter{Integer-Überlauf}


Der Integer-Überlauf ist ein häufiges Problem in vielen Programmiersprachen, das zu schwerwiegenden Fehlern und Sicherheitslücken führen kann. 
Ein Integer-Überlauf tritt auf, wenn eine arithmetische Operation das Maximum (oder Minimum) des für den Datentyp zulässigen Wertebereichs überschreitet.
In vielen Programmiersprachen führt dies zu undefiniertem Verhalten oder unvorhersehbaren Ergebnissen, was potenziell ausnutzbare Schwachstellen zur Folge haben kann \cite{cwe190}.\\
\\
Rust hingegen bietet verschiedene Mechanismen, um Integer-Überläufe sicher zu handhaben und das Verhalten explizit zu definieren \cite[Kapitel 3.2]{rust2023}. 

\begin{lstlisting}[language=Rust, caption= Integer-Überläufe, label=list:overflow_example]
pub fn overflow_how_not(){
    let a: i32 = i32::MAX;
    // panic in release mode due to runtime checks
    // wil wrap in release mode -> defined behaviour
    let b = a + 1;  

    println!("{}", b);
}
pub fn overflow(){
    let a: i32 = i32::MAX;

    // checked addition 
    let checked_sum = a.checked_add(1);
    println!("Checked sum: {:?}", checked_sum);  

    // wrapping addition 
    let wrapping_sum = a.wrapping_add(1);
    println!("Wrapping sum: {}", wrapping_sum);  

    // saturating addition 
    let saturating_sum = a.saturating_add(1);
    println!("Saturating sum: {}", saturating_sum);  

    // overflowing addition 
    let (overflowing_sum, overflowed) = a.overflowing_add(1);
    println!("Overflowing sum: {}, overflowed: {}",
        overflowing_sum, overflowed);  
}
\end{lstlisting}
\cleardoublepage
\noindent
Wie in Beispiel \ref{list:overflow_example} zu sehen, bietet Rust mehrere Möglichkeiten, Integer-Überläufe sicher zu handhaben:

\begin{itemize}
\item \textbf{Überprüfte Addition (checked addition)}: Gibt \texttt{None} zurück, wenn ein Überlauf auftritt.
\item \textbf{Wrappende Addition (wrapping addition)}: Wickelt den Wert bei Überlauf um, entsprechend dem Wertebereich des Datentyps.
\item \textbf{Sättigende Addition (saturating addition)}: Begrenzt den Wert auf den maximalen oder minimalen Wert des Datentyps bei Überlauf.
\item \textbf{Überlaufende Addition (overflowing addition)}: Gibt ein \textit{\gls{tupel}} zurück, bestehend aus dem Ergebnis der Addition und einem booleschen Wert, der anzeigt, ob ein Überlauf aufgetreten ist.
\end{itemize}
