\chapter{Sichere Nebenläufigkeit}

\section{Theoretische Grundlagen}

Nebenläufigkeit ist ein wesentlicher Aspekt moderner Softwareentwicklung, insbesondere bei der Entwicklung von Systemen, die eine hohe Leistung und Reaktionsfähigkeit erfordern. 
Jedoch birgt die Nebenläufigkeit zahlreiche Herausforderungen, insbesondere im Hinblick auf die Sicherheit und Korrektheit des Codes.
Datenrennen sind ein häufiges Problem, das auftritt, wenn mehrere Threads gleichzeitig auf dieselben Daten zugreifen und diese ändern, ohne dass die Zugriffe korrekt synchronisiert sind.\\ 
\\
Rust bietet einen Ansatz zur sicheren Nebenläufigkeit, indem es die Konzepte der Besitzrechte (Ownership) und des Ausleihens (Borrowing) auf Threads anwendet. 
Dies wird durch die Einbindung von Mechanismen wie \texttt{Arc} (Atomic Reference Counting) und \texttt{Mutex} (Mutual Exclusion) erreicht, die sicherstellen, dass Datenrennen verhindert werden, während gleichzeitig eine hohe Leistung erzielt wird.\\
\\
\texttt{Arc} (Atomic Reference Counting) ist ein Thread-sicherer Referenzzähler, der es ermöglicht, dass mehrere Threads gleichzeitig Besitz an den gleichen Daten haben, indem er die Daten im Heap speichert und den Zähler atomar verwaltet. 
Dadurch bleibt der Speicher erhalten, solange mindestens eine Referenz darauf existiert.\\
\\
\texttt{Mutex} (Mutual Exclusion) ist ein Synchronisationsprimitiv, das sicherstellt, dass nur ein Thread zu einem bestimmten Zeitpunkt auf die geschützten Daten zugreifen kann. 
Ein \texttt{Mutex} in Rust liefert einen \texttt{MutexGuard}, der den Zugriff auf die Daten kontrolliert und den \texttt{Mutex} freigibt, sobald der Guard außer Gültigkeit gerät.\\
\\
Durch die Kombination von \texttt{Arc} und \texttt{Mutex} wird sichergestellt, dass veränderliche Daten sicher zwischen mehreren Threads geteilt werden können. 
\texttt{Arc} übernimmt das Zählen der Referenzen, während \texttt{Mutex} den exklusiven Zugriff auf die Daten garantiert. 
Dies funktioniert reibungslos mit Rusts Ownership- und Borrowing-System, da \texttt{Arc} die Besitzrechte an den Daten teilt und \texttt{Mutex} den Zugriff kontrolliert, um Datenrennen zu vermeiden. 
Durch das explizite Anfordern und Freigeben von Sperren mit \texttt{MutexGuard} bleibt der Code dennoch performancestark und sicher.

\cleardoublepage
\section{Praktische Beispiele}

Hier ist ein Beispiel, das zeigt, wie man in Rust sicher Threads erstellt, die gemeinsam einen Zähler erhöhen:

\begin{lstlisting}[language=Rust, caption=Sicheres Erstellen von Threads mit Rust]
use std::sync::{Arc, Mutex};
use std::thread;
use std::time::Duration;
use rand::{thread_rng, Rng}; 

pub fn create_increment_threads(){
    let counter = Arc::new(Mutex::new(0));
    let mut children = vec![];

    for _ in 0..5 {
        let counter_clone = Arc::clone(&counter);
        let child = thread::spawn(move || {
            let mut rng = thread_rng(); 
            // Loop to add to the counter 10 times
            for _ in 0..10 { 
                {
                    let mut num = match counter_clone.lock() {
                        Ok(x) => x,
                        Err(_) => todo!(),
                    };
                    *num += 1;
                } // MutexGuard goes out of scope here, 
                  //releasing the lock

                let sleep_time = rng.gen_range(1..=3); 
                // Sleep for random duration
                thread::sleep(Duration::from_millis(sleep_time)); 
            }
        });
        children.push(child);
    }

    // Wait for all threads to complete
    for child in children {
        child.join().unwrap();
    }

    // Print the result
    println!("Final counter value: {}", *counter.lock().unwrap());
}
\end{lstlisting}
\noindent
In diesem Beispiel wird ein gemeinsamer Zähler von mehreren Threads erhöht. 
Die Verwendung von \texttt{Arc<Mutex<i32>>} stellt sicher, dass nur ein Thread gleichzeitig auf den Zähler zugreifen kann, wodurch Datenrennen verhindert werden.\\
\\
Zum Vergleich zeigt das folgende Beispiel, wie man es \emph{nicht} machen kann, es gibt einen compile error:

\begin{lstlisting}[language=Rust, caption=Unsicheres Erstellen von Threads in Rust]
pub fn how_not_to_create_increment_threads(){
    let mut counter = 0;
    // need to use a reference 
    // otherwise int will be copied for each thread.
    // Explicit mutable reference to `counter`
    let counter_ref = &mut counter; 

    let mut handles = vec![];

    for _ in 0..10 {
        // use the mutable reference in multiple threads
        let handle = thread::spawn(move || {
            // Error: `counter_ref` 
            // cannot be sent safely between threads
            *counter_ref += 1; 
        });
        handles.push(handle);
    }

    for handle in handles {
        handle.join().unwrap();
    }

    println!("Counter: {}", counter);
}
\end{lstlisting}
\noindent
Dieses Beispiel zeigt einen unsicheren Ansatz, bei dem eine mutable Referenz zu einem Zähler in mehrere Threads übergeben wird. 
Dies führt zu einem Kompilierungsfehler, da Rusts Typensystem sicherstellt, dass solche unsicheren Zugriffe verhindert werden.

\section{Vergleich mit anderen Programmiersprachen}

Im Vergleich zu anderen Programmiersprachen wie C++ und Java bietet Rust erhebliche Sicherheitsvorteile. 
In C++ beispielsweise muss der Programmierer manuell sicherstellen, dass Zugriffe auf geteilte Daten korrekt synchronisiert werden, was fehleranfällig und schwierig ist. 
Java bietet eingebaute Synchronisationsmechanismen, aber diese sind oft mit Leistungseinbußen verbunden und weniger explizit in ihrer Nutzung.\\
\\
Rusts Ansatz kombiniert die Leistungsvorteile von C++ mit der Sicherheit und Einfachheit der Synchronisation in Java, bietet jedoch eine explizitere und sicherere Kontrolle über die Nebenläufigkeit durch sein Besitz- und Ausleihmodell.

\section{Sicherheitsvorteile}

Die Sicherheitsvorteile von Rusts Ansatz zur Nebenläufigkeit sind erheblich:

\begin{itemize}
    \item \textbf{Verhinderung von Datenrennen:} Durch die Verwendung von \texttt{Arc} und \texttt{Mutex} stellt Rust sicher, dass Datenrassen auf kompilierbarem Weg verhindert werden.
    \item \textbf{Explizite Synchronisation:} Rust erfordert, dass alle Synchronisationspunkte explizit definiert werden, was die Lesbarkeit und Wartbarkeit des Codes erhöht.
    \item \textbf{Hohe Leistung:} Trotz der zusätzlichen Sicherheit bietet Rust eine hohe Leistung, da es unnötige Synchronisationskosten vermeidet.
\end{itemize}
\noindent

