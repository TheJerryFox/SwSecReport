\section*{Abstract}
\label{sec:abstract}


Dieser Bericht behandelt die zentralen Sicherheitsaspekte der Programmiersprache Rust, die speziell entwickelt wurde, um sichere und effiziente Software zu schreiben. 
Rust kombiniert leistungsstarke Kontrolle über Systemressourcen mit modernen Spracheigenschaften, die viele häufige Sicherheitsprobleme in der Softwareentwicklung verhindern \cite{RustDoc2024}.\\
\\
Zunächst wird die statische Typisierung von Rust erläutert, die durch strikte Typenkontrolle zur Vermeidung typischer Fehler beiträgt. 
Anschließend wird gezeigt, wie Rust Integer-Überläufe verhindert und damit eine häufige Quelle von Sicherheitslücken eliminiert. 
Ein weiterer Schwerpunkt liegt auf den Zero-Cost-Abstraktionen, die es Entwicklern ermöglichen, effizienten und dennoch sicheren Code zu schreiben.\\
\\
Der Bericht beleuchtet auch Rusts Ansätze zur Fehlerbehandlung, die auf das Konzept der Ausnahmen verzichten und stattdessen explizite Fehlerwerte nutzen. 
Besonders hervorzuheben ist Rusts Besitz- und Ausleihmodell, das durch strikte Regeln für Speicherzugriffe und Lebensdauern von Objekten die Sicherheit und Stabilität des Codes gewährleistet. 
Ergänzend dazu wird das Lebensdauer-Tracking vorgestellt, das durch präzise Kontrolle über die Lebensdauer von Referenzen die Gefahr von hängenden Referenzen minimiert.\\
\\
Schließlich wird Rusts Ansatz zur sicheren Nebenläufigkeit untersucht, der Datenrennen (Data Races) und damit verbundene Sicherheitsprobleme verhindert. 
Durch diese Sicherheitsmechanismen bietet Rust eine robuste Basis für die Entwicklung sicherer und zuverlässiger Software, die den Anforderungen moderner Systemprogrammierung gerecht wird.\\
\\
Dieser Bericht gibt einen Überblick über die wichtigsten Spracheigenschaften von Rust und zeigt anhand Beispiele, wie diese in der Praxis angewendet werden können.
