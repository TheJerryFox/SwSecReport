\chapter{Fehlerbehandlung}
Rust bietet einen robusten Ansatz zur Fehlerbehandlung, der ohne Ausnahmen auskommt und somit eine sichere und vorhersehbare Fehlerbehandlung ermöglicht. 
Im Gegensatz zu vielen anderen Programmiersprachen verwendet Rust das \texttt{Result} und \texttt{Option} Typsystem, um Fehler und fehlende Werte explizit zu behandeln.

\section{Theoretische Grundlagen}
In Rust werden Fehler durch die Typen \texttt{Result<T, E>} und \texttt{Option<T>} behandelt. 
\texttt{Result<T, E>} ist ein Typ, der entweder einen Wert vom Typ \texttt{T} oder einen Fehler vom Typ \texttt{E} enthält. 

\section{Praktische Beispiele}
Im Folgenden werden zwei Beispiele gezeigt: eines, das keine sichere Fehlerbehandlung verwendet, und eines, das Rusts robusten Ansatz zur Fehlerbehandlung nutzt.

\begin{lstlisting}[language=Rust, caption=Unsichere Fehlerbehandlung]
use std::fs::File;
use std::io::Read;

pub fn get_file_unsave() {
    // This will panic if there's an error
    let contents = read_file_content_unsave(); 
    println!("File contents: {}", contents);
}

fn read_file_content_unsave() -> String {
    // Risky if file is not found
    let mut file = File::open("example.txt").unwrap();  
    let mut contents = String::new();
    // Panics on error
    file.read_to_string(&mut contents).unwrap();  
    contents
}
\end{lstlisting}
\cleardoublepage
\begin{lstlisting}[language=Rust, caption=Sichere Fehlerbehandlung]
use std::fs::File;
use std::io::Read;

pub fn get_file() {
    match read_file_content() {
        Ok(contents) => println!("File contents: {}", contents),
        Err(e) => println!("Error reading file: {}", e),
    }
}

fn read_file_content() -> Result<String, std::io::Error> {
    let mut file = File::open("example.txt")?;
    let mut contents = String::new();
    file.read_to_string(&mut contents)?;
    Ok(contents)
}
\end{lstlisting}
\noindent
Im ersten Beispiel wird keine Fehlerbehandlung vorgenommen. 
Das Programm wird mit \texttt{unwrap()} gezwungen, den Inhalt der Datei zu lesen, was zum Absturz des Programms führt, falls ein Fehler auftritt. Im zweiten Beispiel wird die \texttt{Result}-Typ verwendet, um Fehler abzufangen und zu behandeln, was zu einem robusteren und sichereren Code führt.\\
\\
\texttt{Unwrap()} gibt von einem Fehlertyp den Wert zurück falls ein Fehler vorliegt geht das Program in eine \textit{\gls{panic}}. 
Mittels \texttt{match} kann überprüft werden, ob das Result \texttt{OK} ist und ein Wert vorliegt oder ob das \texttt{Result} ein Fehler ist.

\section{Vergleich mit anderen Programmiersprachen}
In vielen anderen Programmiersprachen, wie z.B. C++ oder Java, werden Ausnahmen \\ (\textit{\glspl{exception}}) verwendet, um Fehler zu behandeln. 
Rusts Ansatz, Fehler explizit zu behandeln, führt zu einem vorhersebahren Verhalten und fördert den bewussten Umgang mit Fehlern \cite[Kapitel 9]{rust2023}.

\begin{itemize}
    \item \textbf{C++}: Verwendet Ausnahmen, die oft schwer nachzuvollziehen und zu debuggen sind.
    \item \textbf{Java}: Nutzt ebenfalls Ausnahmen, wobei der Entwickler sicherstellen muss, dass alle möglichen Ausnahmen abgefangen werden.
    \item \textbf{Rust}: Verwendet das \texttt{Result} und \texttt{Option} Typsystem, um Fehler explizit zu behandeln.
\end{itemize}

\section{Sicherheitsvorteile}
Rusts Ansatz zur Fehlerbehandlung bietet mehrere Sicherheitsvorteile:

\begin{itemize}
    \item \textbf{Vorhersehbarkeit}: Da Fehler explizit behandelt werden müssen, ist das Verhalten des Programms vorhersehbarer.
    \item \textbf{Vermeidung von Panics}: Durch die Verwendung von \texttt{Result} und \texttt{Option} können viele \textit{\glspl{panic}} vermieden werden, was die Stabilität des Programms erhöht.
    \item \textbf{Robustheit}: Der Code wird robuster, da alle möglichen Fehlerfälle berücksichtigt und behandelt werden, sofern das unsichere \texttt{unwrap()} vermieden wird.
\end{itemize}
\noindent

